%\documentclass[twocolumn,oneside,article,9pt]{memoir}
\documentclass[english,a5paper,oneside, onecolumn,article,9pt]{memoir}

\usepackage[english]{babel}
%\usepackage[ansinew]{inputenc}
%\usepackage[T1]{fontenc}
\usepackage{graphicx}
\usepackage{subcaption}
%\usepackage{amsmath,amssymb,amsthm,bm,mathtools}
%\usepackage{lipsum}
%\usepackage{tabularx}
%\usepackage[draft,english,footnote,nomargin]{fixme}
%\usepackage{microtype}
%\usepackage{lmodern}
%\usepackage{hyperref}
%\usepackage{siunitx}
\usepackage[]{algorithm2e}

\usepackage{hyperref}
\hypersetup{
  colorlinks=true,
  linkcolor=red,          % color of internal links
  citecolor=green,        % color of links to bibliography
  filecolor=magenta,      % color of file links
  urlcolor=cyan           % color of external links
}



% Marginer
\setlrmarginsandblock{20mm}{*}{1}
\setulmarginsandblock{20mm}{*}{1.2}
\checkandfixthelayout[nearest]

% Tillad ubalancerede spalter (at de ikke er lige høje)
\raggedbottom

% Indholdsfortengelse
\setsecnumdepth{subsection}
\maxtocdepth{subsection}

%------------- Signature ----------------------
\newcommand\Signature[4]{
  \bigskip
  \begin{@treecolumnfalse}
    \begin{center}
      \begin{tabular}{
          @{\hspace{.05\textwidth}}
          >{\centering\arraybackslash}p{.4\textwidth}
          @{\hspace{.1\textwidth}}
          >{\centering\arraybackslash}p{.4\textwidth}
		  @{\hspace{.05\textwidth}}
          >{\centering\arraybackslash}p{.4\textwidth}
          @{\hspace{.05\textwidth}}}
        \multicolumn{2}{c}{#1, \today} \\[10ex]
        \hrulefill & \hrulefill \\
        #2 & #3 & #4 \\[10ex]
      \end{tabular}
    \end{center}
  \end{@twocolumnfalse}
}
%----------------------------------------------


% Reimplementer \affiliation
%\makeatletter
%\newcommand{\affiliation}[1]{\gdef\@ffiliation{#1}}
%\newcommand{\@ffiliation}{}
%\renewcommand{\maketitlehookc}{\begin{center}\@ffiliation\end{center}}
%\makeatother

%---------------- Basic info ------------------
\title{\Huge{AI4} \Large{ \\ Liquid State Mahine \\ Implementation and Testing}}
\author{Daniel Tang Hviid \\ Dahvi14@student.sdu.dk}
%\setlength\parindent{0pt}

\pagenumbering{gobble}

\begin{document}
%\twocolumn[
\maketitle

\begin{figure}[h]
    \centering
        \includegraphics[width=0.9\textwidth]{Images/Logo.png}
\end{figure}

\newpage

\begin{centering}
\section*{Abstract}
\end{centering}

Liquid State Machines are a non-differential Neural Network model that contains temporal information with a reservoir hidden layer, that only needs the output layer to be trained.
This paper explains the motive behind implementing a Liquid State Machine, how the theory behind such a model works, goes through the implementation of said model used in this project, and shows qualitative tests of the implementation where individual neurons are investigated.
Finally, this paper shows the Liquid State Machine applied to a sequence prediction problem where the model must correctly predict the next item in the sequence.


\tableofcontents
\newpage

%]
\saythanks

\pagenumbering{arabic}

\chapter{Introduction}

In this paper we go through the motivation behind the project, as well as the theory of the Spiking Neural Network and Reservoir computing, followed by the implementation of a Liquid State Machine in a c++ environment and the tests used to ensure that the implementation works as intended including the prediction of the next item in a repeating sequence.

A Liquid State Machine is a Neural Network model that attempts to more closely resemble a biological brain than other models. It retains temporal information, and is non-differentiable. This makes the model hard to train, which is why this project only looks at training the output layer.



\chapter{Motivation}

In this section we look at why Artificial Neural Networks differ from biological neural networks, and why this is worth emulating and investigating. Then we look at some of the ways they can be modified to better emulate the biology of the brain

Standard neural networks are based on mathematical models. These models share little with the functions of the brain, as they are a much oversimplified version of the biological network that constitutes brainmatter, which is done in order to take advantage of differential mathematics.
As biological neural networks have so far unparallelled computational powers and uses, this report is motivated by the idea that these biological networks are worth emulating in search of more powerful and general neural models.

There are several differences between biological neural networks, such as a human brain, and artifical neural networks, such as a feed forward network. These include the temporal states, the non-differential signals that the brain cells use to communicate with which are spiking voltages, activation functions that try to mimic the sodium-pumps of the brain, as well as the overlying structure of the network seeing that nature rarely conforms itself into neat layers and lines.

Looking only at the first three items on this list, there is an artifical neural network model that does take these into considerationg and as such better immitates the workings of the brain. This is the spiking neural network model, which uses the concept of spike trains instead of linear transformations to process the input information. These are said to have higher expressional power than standard neural networks, but in return require more computational power to run.
The last item is structure. A biological brain is not made to be uni-directional, such as a feed-forward neural network, but instead is much better expressed as a reservoir computing framework. When one combines these two extensions of the feed-forward network, one arrives at what is called a Liquid State Machine.

\section{Theory}

In this section we look at the teory behind Spiking neural networks, reservoir computing, and Liquid State Machines, assuming that the reader is familiar with Artificial neural networks.

\subsection{Spiking Neural Network}

The main difference between a Spiking Neural Network and a non-Spiking Neural Network is the way it processes the weighted input sum in order to find the output. For a non-Spiking NN, this is the activation function, while a Spiking NN uses a model that stores information from activation to activation. This main difference is what provides the Spiking NN with temporal information that permeates through the network.

Several such models exist, but only the Leaky Integrate-and-fire model \cite{Leaky} will be explained and used in this project.

The Leaky Integrate-and-fire model attempts to emulate the membrane potential and voltage spikes of a biological neuron. The biological neuron uses two different ion channels in order to build up potential and then dump it as a spike, which is modelled using the following equation, from \cite{leaky}:

\begin{equation}
\tau_m\frac{dv}{dt} = -v(t) + RI(t)
\end{equation}

Where [$\tau_m$] is the time constant, [$v(t)$] is the membrane potential at a given time t, R is the resistance of the membrane, and $I(t)$ is the integral of the input current.
There are three different parts to this. The first is the change in membrane potential; This change in potential is the main part of the equation, as it is what we wish to find at each time step. The second part is the ``leaky'' part of the Leaky Integrate-and-fire, where [$v(t)$] is the membrane potential at a given time t. At each timestep a part of the stored potential is lost, which is proportional in size with the stored potential. The last part is the integral of the weighted inputs, where R is the resistance of the membrane, and $I(t)$ is the integral of the input current, as with a non-Spiking NN. The time constant [$\tau_m$] then scales all of this to fit with the time between each activation.

\subsection{Reservoir computing}

A reservoir Neural Network is an extension to an ANN where the hidden layer is made in such a way that instead of having several layers that feed to each other in order, the individual neurons are simply randomly connected to each other with random weights.

\section{Implementation}

In this section we look into how the Liquid State Machine was implemented in a code environment, and which choices were made along the way.
The code implementations of this project used C++ as a language, and generally used Object Oriented design philosophy, aiming to be modular and easy to use.

The core of the implementation is the hidden layer of the Liquid State Machine, from now on refered to as the Pool. The Pool have input and output neurons, that are interfaced with by the input and output layers. %(image needed)
The pool is implemented a 3 dimensional tensor of pointers to individual neurons. These neurons each contain a set of parameters, the important ones listed below:

\begin{itemize}
\item Input
\item Internal potential
\item Output
\item Synapses
\item Weights
\end{itemize}

As well as a list of constants pertaining to the specific neuron model used, see \ref{leaky}.

The secondary part of the implementation is the input and output layers. In this implementation, these are implemented as standard feed forward networks. This method was chosen in order to facilitate the back propogation capabilities of a feed forward network in the output layer, which was then reused as the input layer for simplicity. As the two models are not compatible, the data needs to be converted from one to another at the transitions. The input transition is simple, it just adds to the input parameter of the Pool inputs, but the output transition is a bit more tricky.
Because the implementation allows for two different kinds of activations, one at each time step and one that sums over a single time unit, two different transitions needs to be implemented.
On the timestep activation the internal potential of the Pool outputs are simply given to the output layer as inputs, unless the neuron had just fired, in which case the output is 1.
The timeunit activation instead takes the mean of the activations for the duration of the activation. Because of this, both activation methods then provide a number below 1, which allows for consistency when training the outputlayer.

\subsection{Algorithm}

Now that the elements are accounted for, the algorithm that is run each time the LSM is activated can be explained.

At each step the first thing that needs to be done is running the input layer. This yields a set of values that are then provided to the Pool input layer.
The next step is activating all of the neurons whose internal potential exceeded the threshold potential last timestep.
The third step is to check the internal potential of each neuron, to see which are going to fire next timestep.
The last step is to gather the output from the Pool outputs.


\chapter{Testing}

This section documents the tests used to ensure that the implementation of Liquid State Machine is functional, and to highlight certain properties of both the implementation and the network model itself.

\section{Neuron model}

There exists several neural models that can be used for a Liquid State Machine, or a Spiking Neural Network. The selected model is the Leaky-Integrate-and-Fire model \ref{sec:leaky}, as it is simple, not computationally heavy, and expressive enough for this project.

In order to test this model, a baseline set of parameters must first be selected. The baseline parameters can be found in table \ref{tab:neuron}. For these parameters, the model behaves as can be seen in figure \ref{fig:model1} which is clearly a smooth membrane potential curve followed by a spike as the potential reaches the spiking threshold. In figure \ref{fig:model2}, the parameters have been changed to allow a random input between zero and one, thus illustrating how a neuron would act when it's input is affected by several synapses that fire independent of each other. It is clear from these two images, as the neuron fires in exactly the same timestep, that the mean input is relevant, even if the input is not consistent.

\begin{table}[h]
\centering
\begin{tabular}{cccc}
Resistance & Time constant & Resting potential & Spiking potential\\
2.2 & 10 & 0 & 1  
\end{tabular}
\caption{Table of parameters used when testing neuron activity.}
\label{tab:neuron}
\end{table}

\begin{figure}
    \centering
    \begin{subfigure}[b]{0.3\textwidth}
        \includegraphics[width=\textwidth]{Images/slow.png}
        \caption{The neuron model using the baseline parameters and constant input of 0.5.}
        \label{fig:implementation_test}
    \end{subfigure}
    \begin{subfigure}[b]{0.3\textwidth}
        \includegraphics[width=\textwidth]{Images/randomInput.png}
        \caption{The neuron model using the baseline parameters and random input with a mean of 0.5.}
        \label{fig:implementation_test}
    \end{subfigure}
    \begin{subfigure}[b]{0.3\textwidth}
        \includegraphics[width=\textwidth]{Images/fast.png}
        \caption{The neuron model using double resistance, and random input with a mean of 0.5.}
        \label{fig:implementation_test}
    \end{subfigure}
    \caption{Tests of the neuron model.(a) shows the ideal potential curve followed by a spike. (b) shows how the potential curve caused by a randomized input can be used in place of the ideal curve to get the same frequency of the spike. (c) shows how doubling the resistance parameter does not double the fireing rate of the neuron.}
    \label{fig:implementation_test}
\end{figure}

If one instead changes the parameters so that the resistance is twice of what it was in the baseline, and sets the input to a random number between zero and one each timestep, the result, as seen in figure \ref{fig:fast}, the neuron fires three times in the same timestep, showing how the resistance is not proportional to the fire rate of the neuron.

\section{Propagation}

In a functional Liquid State Machine, the activation of a neuron in the hidden layer will spread to nearby neurons like ripples in a pond. In order to test this, one can activate a single neuron and observe this behavior through the activation of the connected neurons. One way of performing this observation is by setting the neuron parameters in such a way that a single received pulse is enough to cause an activation, and then activating a single neuron in the reservoir. The expected behavior is a cascade effect where each activation will activate several other neurons, eventually activating each connected neuron in the reservoir.

By visualizing the activity of the network through repeated activations, this is easy to test. In figure \ref{fig:implementation_test}, it is clear that a single reservoir neuron is activated after the first timestep, and that after ten timesteps all connected neurons have been activated. In this figure it is also clear that three adjacent neurons, which is the number of connections each neuron have, are activated in the second timestep, as is expected from a functional Liquid State Machine.

\begin{figure}
    \centering
    \begin{subfigure}[b]{0.225\textwidth}
        \includegraphics[width=\textwidth]{Images/Reservoir_test_Activation_number_1.png}
        \caption{The test network after a single timestep.}
    \end{subfigure}
    \begin{subfigure}[b]{0.225\textwidth}
        \includegraphics[width=\textwidth]{Images/Reservoir_test_Activation_number_2.png}
        \caption{The test network after two timesteps.}
    \end{subfigure}
    \begin{subfigure}[b]{0.225\textwidth}
        \includegraphics[width=\textwidth]{Images/Reservoir_test_Activation_number_5.png}
        \caption{The test network after five timesteps.}
    \end{subfigure}
     \begin{subfigure}[b]{0.225\textwidth}
        \includegraphics[width=\textwidth]{Images/Reservoir_test_Activation_number_10.png}
        \caption{The test network after ten timesteps.}
    \end{subfigure}
    \caption{Test of the propagation of the network. Every neuron has three synapses which are adjacent to it, and each spike will activate all three synases. After the first timestep, only the selected neuron is active. In the second timetep, it has spiked and is therefore inactive while the three neurons it has synapses to are active. After five timesteps a large proportion of the neurons are active, and after ten all connected neurons are active.}
    \label{fig:implementation_test}
\end{figure}

\section{Reservoir activity}

The reservoir parameters are as important as the neuron parameters. There are several parameters that can be tuned and modified, and in this section the number of connections as well as the proportion of exitory to inhibitory neurons are tuned. The rest of the parameters are kept constant, as well as the neuron parameters, which can be found in table \ref{tab:reservoir}.

\begin{table}[h]
\centering
\begin{tabular}{cccc}
Resistance & Time constant & Resting potential & Spiking potential\\
2.2 & 10 & 0 & 1
\end{tabular}
\begin{tabular}{ccc}
Min weight & Max weight & Max synapse length\\
0 & 1 & 2
\end{tabular}
\begin{tabular}{cc}
Input neurons & Connections per input neuron\\
3 & 10
\end{tabular}
\caption{Table of parameters used for testing network activity when testing different amounts of connectors per neuron.}
\label{tab:reservoir}
\end{table}

 Having too few connections will result in low amount of activity that won't spread beyond the input neurons. Conversely, having too many connections will result in a cascade of activity that results in all, or most, neurons activating every timestep. Both of these options are equally useless, as they provide essentially constant output.

First, to test the impact of the number of connections that each neuron have on the level of activity of the reservoir, two different amounts of neuron connections are tested. The reservoir after one hundred timestept with six and seven connections per neuron can be seen in figure \ref{fig:reservoir_6} and \ref{fig:reservoir_7}. As can be seen in the first image, six connections are enough to provide a good level of activity throughout the network, without having a lot of the neurons activate each timestep. On the other hand, having seven connections is enough to cause a run-away reaction that activates almost all the neurons each timestep. Clearly, there is a fine balance between having enough connections, and having too many.

A way of correcting this is by including inhibitory neurons. In figure \ref{fig:reservoir_7_in} we see how 15\% inhibitory neurons brought the level of activity back down, and in figure \ref{fig:reservoir_10_in}.

\begin{figure}
    \centering
    \begin{subfigure}[b]{0.4\textwidth}
        \includegraphics[width=\textwidth]{Images/pool_6_100.png}
        \caption{The test network after a hundred timesteps with 6 connections per neuron.}
    \label{fig:reservoir_6}
    \end{subfigure}
    \begin{subfigure}[b]{0.4\textwidth}
        \includegraphics[width=\textwidth]{Images/pool_7_100.png}
        \caption{The test network after a hundred timesteps with 7 connections per neuron.}
    \label{fig:reservoir_7}
    \end{subfigure}
    \begin{subfigure}[b]{0.4\textwidth}
        \includegraphics[width=\textwidth]{Images/pool_7_100_s.png}
        \caption{The test network after a hundred timesteps with 7 connections per neuron, with inhibitors.}
    \label{fig:reservoir_7_in}
    \end{subfigure}
     \begin{subfigure}[b]{0.4\textwidth}
        \includegraphics[width=\textwidth]{Images/pool_10_100_s.png}
        \caption{The test network after a hundred timesteps with 10 connections per neuron, with inhibitors.}
    \label{fig:reservoir_10_in}
    \end{subfigure}
    \caption{Spikes and membrane potential after a hundred timesteps with different amounts of connections per neuron and inhibitors.}
\end{figure}

\section{Training}
















\section{Discussion}

In this section we discuss the Liquid State Machine, more specifically the choises taken when implementing it, the troubles of tuning it, and the results of the tests.

When implementing a neural network, one would usually make a set of matrices and tensors that contain the inputs, outputs, activations and weights. In this implementation, however, a graph implementation with nodes for neurons was used. This impacted the computational time of the network, but also allowed for easier bugfixing and faster implementation. This tradeoff was made because of the limited scope of the project, as well as the relatively small problem it was used on. This allowed for the implementation of the Neuron class, which made it much easier to implement it in a modular fashion.

Tuning a Liquid State Machine is a tedious process, as a small change in parameters will make the network lose a lot of accuracy. In particular the number of connections, neuron resistance, resting potential, proportion of inhibatory neurons, and number of synapses per input neurons. These parameters should be tuned quite gently, as they will make the network either almost inactive or start a chain reaction.

\section{Conclusion}

Making an artificial neural network that more closely resemble the human brain than perceptrons and similar models require spiking models as well as more reservoir-like layers. The Liquid State Machine fullfills this, as it is reservoir computing that uses spiking neuron models. This project implemented this model in a c++ environment, keeping the implementation modular to allow for easy testing. The implementation was then tested on a relatively easy case, where the Liquid State Machine was used to predict the next letter in a repeating sequence, with an accuracy of 100\%. As such, the goal of implementing and testing the Liquid State Machine was achieved.

\begin{thebibliography}{9}

\bibitem{leaky}
 Hélène Paugam-Moisy and Sander Bohte\\
 Computing with Spiking Neuron Networks\\
 \textit{pdfs.semanticscholar.org/8997/95d8e3c07976f62149eb79612e06231d6aee.pdf}

\end{thebibliography}





\end{document}

